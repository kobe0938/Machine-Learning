\item \subquestionpoints{5} \textbf{Cocktail Party Problem}

For this question you will implement the Bell and Sejnowski ICA algorithm, but
assuming a Laplace source (as derived in part-b), instead of the Logistic distribution
covered in class. The file \texttt{src/ica/mix.dat} contains the input data which consists of a matrix
with 5 columns, with each column corresponding to one of the mixed signals
$x_i$. The code for this question can be found in \texttt{src/ica/ica.py}.

Implement the \texttt{update\_W} and \texttt{unmix} functions in \texttt{src/ica/ica.py}.

You can then run \texttt{ica.py} in order to split the mixed audio into its components.
The mixed audio tracks are written to \texttt{mixed\_i.wav} in the output folder.
The split audio tracks are written to \texttt{split\_i.wav} in the output folder.

To make sure your code is correct, you should listen to the
resulting unmixed sources.  (Some overlap or noise in the sources may be present,
but the different sources should be pretty clearly separated.)

\textbf{Submit the full unmixing matrix $W$ (5$\times$5) that you obtained, by including the W.txt the code outputs along with your code.}

If you implemention is correct, your output \texttt{split\_0.wav} should sound similar to the file \texttt{correct\_split\_0.wav} included with the source code.

Note: In our implementation, we {\bf anneal} the learning rate $\alpha$
(slowly decreased it over time) to speed up learning. In addition to using the variable
learning rate to speed up convergence, one thing that we also do is
choose a random permutation of the training data, and running stochastic
gradient ascent visiting the training data in that order (each of the
specified learning rates was then used for one full pass through the data).

