\item \subquestionpoints{5} \textbf{Coding problem: ideal (fully observed) case}

First we will consider the hypothetical (and uninteresting) case, where we have access to the true
$t$-labels for training. In \texttt{src/posonly/posonly.py}, write a logistic
regression classifier that uses $x_1$ and $x_2$ as input features, and train it
using the $t$-labels. We will ignore the $y$-labels for this part. Output the
trained model's predictions on the \textbf{test set} to the file specified in the code.

Create a plot to visualize the test set with $x_1$ on the horizontal axis and $x_2$ on
the vertical axis. Use different symbols for examples $x^{(i)}$ with true label $t^{(i)} = 1$
than those with $t^{(i)} = 0$. On the same figure, plot the decision boundary obtained
by your model (i.e, line corresponding to model's predicted probability = 0.5) in red color. Include
this plot in your writeup.


